\everymath{\displaystyle}
\documentclass{beamer}
% \documentclass[handout]{beamer}

%\usepackage[pdftex]{color,graphicx}
\usepackage{amsmath,amssymb,amsfonts}

\mode<presentation>
{
  % \usetheme{Darmstadt}
  % \usetheme[hideothersubsections]{Hannover}
  % \usetheme[hideothersubsections]{Goettingen}
  \usetheme[hideothersubsections, right]{Berkeley}

  \usecolortheme{seahorse}
  % \usecolortheme{dolphin}
  \usecolortheme{rose}
  % \usecolortheme{orchid}

  \useinnertheme[shadow]{rounded}

  \setbeamercovered{transparent}
  % or whatever (possibly just delete it)
}

\mode<handout>{
  \setbeamercolor{background canvas}{bg=black!5}
  \usepackage{pgfpages}
  \pgfpagesuselayout{4 on 1}[a4paper,border shrink=5mm, landscape]
}

\usepackage[brazilian]{babel}
% or whatever

% \usepackage[latin1]{inputenc}
\usepackage[utf8]{inputenc}
% or whatever

\usepackage{times}
%\usepackage[T1]{fontenc}
% Or whatever. Note that the encoding and the font should match. If T1
% does not look nice, try deleting the line with the fontenc.


\title%[] % (optional, use only with long paper titles)
{Introdução}

\subtitle
{Conhecimento} % (optional)

\author%[] % (optional, use only with lots of authors)
{Felipe Figueiredo}% \and S.~Another\inst{2}}
% - Use the \inst{?} command only if the authors have different
%   affiliation.

\institute[INTO] % (optional, but mostly needed)
{Instituto Nacional de Traumatologia e Ortopedia
}
  % \inst{1}%
  % Department of Computer Science\\
  % University of Somewhere
  % \and
  % \inst{2}%
  % Department of Theoretical Philosophy\\
  % University of Elsewhere}
% - Use the \inst command only if there are several affiliations.
% - Keep it simple, no one is interested in your street address.

\date%[] % (optional)
{}

% \subject{Talks}
% This is only inserted into the PDF information catalog. Can be left
% out. 



% If you have a file called "university-logo-filename.xxx", where xxx
% is a graphic format that can be processed by latex or pdflatex,
% resp., then you can add a logo as follows:

\pgfdeclareimage[height=1.6cm]{university-logo}{../logo}
\logo{\pgfuseimage{university-logo}}



% Delete this, if you do not want the table of contents to pop up at
% the beginning of each subsection:
\AtBeginSubsection[]
%\AtBeginSection[]
{
  \begin{frame}<beamer>{Sumário}
    \tableofcontents[currentsection,currentsubsection]
  \end{frame}
}


% If you wish to uncover everything in a step-wise fashion, uncomment
% the following command: 

\beamerdefaultoverlayspecification{<+->}


\begin{document}

\begin{frame}
  \titlepage
\end{frame}

\begin{frame}{Sumário}
  \tableofcontents
  % You might wish to add the option [pausesections]
\end{frame}


%% Template
% \section{}

% \subsection{}

% \begin{frame}{}
%   \begin{itemize}
%   \item 
%   \end{itemize}
% \end{frame}

% \begin{frame}
%   \begin{columns}
%     \begin{column}{5cm}
%     \end{column}
%     \begin{column}{5cm}
%     \end{column}
%   \end{columns}
% \end{frame}

% \begin{frame}{}
%   \includegraphics[height=0.4\textheight]{file1}
%   \includegraphics[height=0.4\textheight]{file2}
%   \includegraphics[height=0.4\textheight]{file3}
%   \begin{figure}
%     \caption{}
%   \end{figure}
% \end{frame}

% \begin{frame}{}
%   \begin{definition}
%   \end{definition}
%   \begin{example}
%   \end{example}
%   \begin{block}{Exercício}
%   \end{block}
% \end{frame}

\section{A disciplina}

\subsection{Apresentação}

\begin{frame}{Docente}
\begin{block}{Nome}
    \href{https://www.linkedin.com/pub/felipe-figueiredo/13/896/979}{Felipe Figueiredo}
  \end{block}

\bigskip

\begin{block}{Email}
  \href{mailto:prof.felipefigueiredo@gmail.com}{prof.felipefigueiredo@gmail.com}
\end{block}

\bigskip

\begin{block}{Página}
  \href{http://sites.google.com/site/proffelipefigueiredo}{http://sites.google.com/site/proffelipefigueiredo}
\end{block}

\end{frame}

\subsection{Disciplina}

\begin{frame}{Objetivos}

  \begin{block}{Principal}
    \begin{itemize}
    \item Redação de um projeto
    \end{itemize}
  \end{block}

  \begin{block}{Secundários}
    Oferecer uma introdução suave a
    \begin{itemize}
    \item Metodologia científica
    \item Redação científica
    \end{itemize}
  \end{block}

\end{frame}

\begin{frame}{Bibliografia}
  \begin{itemize}
  \item {\bf Metodologia do Trabalho Científico: Métodos e Técnicas da Pesquisa e do Trabalho Acadêmico} (PRODANOV, Cleber Cristiano; de FREITAS, Ernani Cesar, 2013).
  \item Outros materiais online linkados na página da disciplina.
  \end{itemize}
\end{frame}

\begin{frame}{O que esperar da disciplina}
  \begin{enumerate}
  \item<2,7> {Introdução: conhecimento} (cap 3)
  \item<2,7> {Métodos científicos} (cap 4)
  \item<3,7> {Problema, Hipóteses, variáveis} (caps 6,7)
  \item<3,7> {EDA}
  \item<3,7> {Etapas, Planejamento, Fracassos} (cap 8)
  \item<4,7> \alert<7>{Tópicos de escrita científica} (Gopen, Swan)
  \item<5,7> {Projeto I - Projeto e relatório} (cap 10)
  \item<5,7> {Projeto II - Dissertação e artigos} (caps 11,12)
  \item<5,7> {Revisão bibliográfica e resumo} (cap 2)
  \item<5,7> {Citações, Referências e Plágio} (cap 13)
  \item<6,7> {Tópicos de busca bibliográfica}
  \item<6,7> \alert<7>{Indicadores em Ciência} (Hirsch)
  \item<1-> Seminários
  \item<1-> Seminários
  \item<1-> Seminários
  \end{enumerate}
  \uncover<7>{ Os itens em vermelho indicam entregas de avaliação.}
\end{frame}

\begin{frame}{Avaliação}
  \begin{itemize}
  \item Escrita de um projeto {\em ad-hoc}.
    \begin{itemize}
    \item<2-> Proposta inicial (sumário)
    \item<3-> Texto final, formatado no padrão ABNT (em PDF).
    \item<3-> Rascunho da apresentação da defesa (em PDF).
    \end{itemize}
  \item<3-> Seminário de defesa do projeto.
  \end{itemize}
\end{frame}

\section{Conhecimento}

\subsection{Conceitos preliminares}

% \subsection{Verdade}

\begin{frame}{Verdade}
  \begin{columns}
    \begin{column}{6cm}<1-> ``{\em When you have eliminated the
        impossible, whatever remains, however improbable, must be the
        truth.}'' Sherlock Holmes
    \end{column}
    \begin{column}{4cm}<1->
      \includegraphics[height=0.4\textheight]{Intro/truth}
    \end{column}
  \end{columns}
\end{frame}

% \subsection{Evidências}

\begin{frame}{Evidências}
  ``{\em It is a capital mistake to theorize before one has
    data. Insensibly one begins to twist facts to suit theories,
    instead of theories to suit facts.}'' Sherlock Holmes
\end{frame}

% \subsection{Dados, Informação e Conhecimento}

\begin{frame}{Dados, Informação e Conhecimento}
  \begin{itemize}
  % \item ``Evidence-based medicine (EBM) is the process of
  %   systematically reviewing, appraising and using clinical research
  %   findings to aid the delivery of optimum clinical care to
  %   patients''
  \item Dados: elementos, códigos ou símbolos quantificáveis que são
    coletados em um experimento.
  \item Informação: agregação e interpretação de dados
  \item Conhecimento: agregação de um corpo de informações que tem
    significado e aplicabilidade prática
  \end{itemize}
\end{frame}


\begin{frame}{Dados, Informação e Conhecimento}
  \begin{center}
    \includegraphics[height=0.9\textheight]{Intro/Knowledge_pyramid}
  \end{center}
  % \begin{figure}
  %   \caption{}
  % \end{figure}
\end{frame}

% \begin{frame}{Dados, Informação e Conhecimento}
%   \begin{example}
%     Dados do mercado financeiro: preço das ações da Petrobrás hoje
%   \end{example}
% \end{frame}

\subsection{Tipos de conhecimento}

\begin{frame}{Conhecimento Filosófico}
  \begin{itemize}
  \item Valorativo (parte de hipóteses que não podem ser observadas)
  \item Racional (enunciados logicamente correlacionados)
  \item Sistemático
  \item Não verificável (conclusões não podem ser confirmadas nem
    refutadas)
  \item Infalível e exato (razão pura)
  \end{itemize}
\end{frame}

\begin{frame}{Conhecimento Teológico}
  \begin{itemize}
  \item Valorativo
  \item Inspiracional
  \item Sistemático
  \item Não verificável
  \item Infalível e exato (divino)
  \end{itemize}
\end{frame}

\begin{frame}{Conhecimento Popular}
  \begin{itemize}
  \item Valorativo (estados de ânimo e emoções)
  \item Reflexivo (não pode ser generalizado)
  \item Assistemático
  \item Verificável (no cotidiano)
  \item Falível e inexato (percepções do dia a dia)
  \end{itemize}
\end{frame}

\begin{frame}{Conhecimento Científico}
  \begin{itemize}
  \item Factual
  \item Contingente (\alert{experimento} ao invés de razão pura)
  \item Sistemático 
  \item \alert{Verificável}
  \item Falível (não é definitivo)
  \item Aproximadamente exato (novos dados podem derrubar teorias
    anteriores)
  \end{itemize}
\end{frame}

\begin{frame}{Classificação da Ciência}
  \begin{center}
    \includegraphics[width=\textwidth]{Intro/ciencias}
  \end{center}
\end{frame}


\subsection{Resumo}

\begin{frame}{Conhecimento Científico x Senso Comum}
  \begin{itemize}
  \item Um mesmo objeto ou fenômeno \alert{pode} ser observado por
    qualquer tipo de conhecimento
  \item diferença: forma de observação
  \item Ciência
    \begin{itemize}
    \item reprodutível
    \item acúmulo incremental
    \item autoavaliação ou correção
    \end{itemize}
  \end{itemize}
\end{frame}

\begin{frame}{A diferença}
  \begin{columns}
    \begin{column}{5cm}
      \begin{center}
        \includegraphics[width=0.55\textwidth]{Intro/the_difference1}
      \end{center}
    \end{column}
    \begin{column}{5cm}
      \begin{center}
        \begin{center}
          \includegraphics[width=\textwidth]{Intro/the_difference2}
        \end{center}
      \end{center}
    \end{column}
  \end{columns}

https://xkcd.com/242/
\end{frame}

\begin{frame}{Exercício}
Escreva um método detalhado para pregar um prego em um sabão

  \begin{center}
    \includegraphics[width=0.8\textwidth]{Intro/pregomartelo}
  \end{center}
\end{frame}

\end{document}
