\everymath{\displaystyle}
\documentclass{beamer}
% \documentclass[handout]{beamer}

%\usepackage[pdftex]{color,graphicx}
\usepackage{amsmath,amssymb,amsfonts}

\mode<presentation>
{
  % \usetheme{Darmstadt}
  % \usetheme[hideothersubsections]{Hannover}
  % \usetheme[hideothersubsections]{Goettingen}
  \usetheme[hideothersubsections, right]{Berkeley}

  \usecolortheme{seahorse}
  % \usecolortheme{dolphin}
  \usecolortheme{rose}
  % \usecolortheme{orchid}

  \useinnertheme[shadow]{rounded}

  % \setbeamercovered{transparent}
  \setbeamercovered{invisible}
  % or whatever (possibly just delete it)
}

\mode<handout>{
  \setbeamercolor{background canvas}{bg=black!5}
  \usepackage{pgfpages}
  \pgfpagesuselayout{4 on 1}[a4paper,border shrink=5mm, landscape]
}

\usepackage[brazilian]{babel}
% or whatever

% \usepackage[latin1]{inputenc}
\usepackage[utf8]{inputenc}
% or whatever

\usepackage{times}
%\usepackage[T1]{fontenc}
% Or whatever. Note that the encoding and the font should match. If T1
% does not look nice, try deleting the line with the fontenc.


\title%[] % (optional, use only with long paper titles)
{Reprodutibilidade de estudos, e indicadores na Ciência}

\subtitle
{} % (optional)

\author%[] % (optional, use only with lots of authors)
{Felipe Figueiredo}% \and S.~Another\inst{2}}
% - Use the \inst{?} command only if the authors have different
%   affiliation.

\institute[INTO] % (optional, but mostly needed)
{Instituto Nacional de Traumatologia e Ortopedia
}
  % \inst{1}%
  % Department of Computer Science\\
  % University of Somewhere
  % \and
  % \inst{2}%
  % Department of Theoretical Philosophy\\
  % University of Elsewhere}
% - Use the \inst command only if there are several affiliations.
% - Keep it simple, no one is interested in your street address.

\date%[] % (optional)
{}

% \subject{Talks}
% This is only inserted into the PDF information catalog. Can be left
% out. 



% If you have a file called "university-logo-filename.xxx", where xxx
% is a graphic format that can be processed by latex or pdflatex,
% resp., then you can add a logo as follows:

\pgfdeclareimage[height=1.6cm]{university-logo}{../logo}
\logo{\pgfuseimage{university-logo}}



% Delete this, if you do not want the table of contents to pop up at
% the beginning of each subsection:
\AtBeginSubsection[]
%\AtBeginSection[]
{
  \begin{frame}<beamer>{Sumário}
    \tableofcontents[currentsection,currentsubsection]
  \end{frame}
}


% If you wish to uncover everything in a step-wise fashion, uncomment
% the following command: 

% \beamerdefaultoverlayspecification{<+->}


\begin{document}

\begin{frame}
  \titlepage
\end{frame}

\begin{frame}{Sumário}
  \tableofcontents
  % You might wish to add the option [pausesections]
\end{frame}


%% Template
% \section{}

% \subsection{}

% \begin{frame}{}
%   \begin{itemize}
%   \item 
%   \end{itemize}
% \end{frame}

% \begin{frame}
%   \begin{columns}
%     \begin{column}{5cm}
%     \end{column}
%     \begin{column}{5cm}
%     \end{column}
%   \end{columns}
% \end{frame}

% \begin{frame}{}
%   \includegraphics[height=0.4\textheight]{file1}
%   \includegraphics[height=0.4\textheight]{file2}
%   \includegraphics[height=0.4\textheight]{file3}
%   \begin{figure}
%     \caption{}
%   \end{figure}
% \end{frame}

% \begin{frame}{}
%   \begin{definition}
%   \end{definition}
%   \begin{example}
%   \end{example}
%   \begin{block}{Exercício}
%   \end{block}
% \end{frame}

\section{Apresentação}

\subsection{O docente e material online}

\begin{frame}{Docente}
  \begin{block}{Nome}
    Felipe Figueiredo
  \end{block}
  \begin{block}{Email}
    \url{prof.felipefigueiredo@gmail.com}

    \bigskip
    \small
    \alert{Atenção:} Salve-o como contato e use o endereço salvo, para mitigar a chance de {\bf estravio}!
    \footnote{O endereço felipefigueiredo@gmail.com {\bf não é meu}!}
  \end{block}
\end{frame}

\begin{frame}{Material online}
  Todo o material didático será disponibilizado na página da disciplina, que fica no Site abaixo
  \begin{block}{Site (http / https)}
    \small
    \url{sites.google.com/site/proffelipefigueiredo/}
  \end{block}
  % Adicionalmente, avisos importantes podem ser divulgados no blog
  % \begin{block}{Blog}
  %   \small
  %   \url{http://proffelipefigueiredo.blogspot.com.br/}
  % \end{block}

  \bigskip
  O endereço não é de fácil memorização, portanto uma busca no Google é o melhor caminho.

  \bigskip
  Você pode procurar pelo meu nome (Felipe Figueiredo)

  \bigskip
  \bigskip
  Porém...
\end{frame}

\begin{frame}{Google: felipe figueiredo}
  \begin{center}
    \includegraphics[height=.7\textheight]{Intro/felipefigueiredo-not1}
    \begin{exampleblock}{}
      \begin{center}
        {\bf Não sou Historiador}
      \end{center}
    \end{exampleblock}
  \end{center}
\end{frame}

\begin{frame}{Google: dr felipe figueiredo}
  \begin{center}
    \includegraphics[height=.7\textheight]{Intro/felipefigueiredo-not2}
    \begin{exampleblock}{}
      \begin{center}
        {\bf Não sou Psiquatra ou Ortopedista}
      \end{center}
    \end{exampleblock}
  \end{center}
\end{frame}

\begin{frame}{Google: prof felipe figueiredo}
  \begin{center}
    \includegraphics[height=.7\textheight]{Intro/felipefigueiredo}
    \begin{exampleblock}{}
      \begin{center}
        Este que vos fala, a seu dispor.
      \end{center}
    \end{exampleblock}
  \end{center}
\end{frame}

\begin{frame}{Material online}
  \begin{center}
    \includegraphics[height=.7\textheight]{Intro/pg-disciplina}
    \begin{exampleblock}{}
      \begin{center}
        PDFs das aulas, artigos, livros, vídeos, exercícios...
      \end{center}
    \end{exampleblock}
  \end{center}
\end{frame}

\subsection{A disciplina}

\begin{frame}{Objetivos de aprendizagem}

  \begin{block}{Principal}
    \begin{itemize}
    \item Redação de um projeto
    \end{itemize}
  \end{block}

  \begin{block}{Secundários}
    Oferecer uma introdução suave a
    \begin{itemize}
    \item Metodologia científica
    \item Redação científica
    \end{itemize}
  \end{block}

\end{frame}

\begin{frame}{Bibliografia}
  \begin{block}{Livro texto}
    {\bf Metodologia do Trabalho Científico: Métodos e Técnicas da Pesquisa e do Trabalho Acadêmico} (PRODANOV, Cleber Cristiano; de FREITAS, Ernani Cesar, 2013).
  \end{block}
  \begin{itemize}
  \item Outros materiais online linkados na página da disciplina.
  \end{itemize}
\end{frame}

\begin{frame}{O que esperar da disciplina}
  \tiny
  \begin{enumerate}
  \item<2,6> {Reprodutibilidade de estudos, e indicadores na Ciência (Hirsch,2005; Garfield, 2006)}
  \item<2,6> {Métodos científicos}
  \item<2,6> {Tópicos de escrita científica} (Gopen e Swan, 1990)
  \item<2,6> {Problema, Hipótese, variável}
  \item<2,6> {Tópicos em formulação de hipóteses}
  \item<2,6> \alert<6>{Etapas da Pesquisa (Planejamento e eventuais fracassos)}
  \item<3,6> {Estrutura do Projeto I - Conteúdo}
  \item<3,6> {Estrutura do Projeto II – Forma}
  \item<3,6> {Citações, Referências e Plágio}
  \item<4,6> {Revisão bibliográfica para a Introdução e Resumo}
  \item<4,6> {Tópicos de Busca Bibliográfica (Google-fu, {\em et al})}
  \item<5,6> \alert<6>{Tópicos em Estratégias de Apresentação}
  \item<1-> \alert<6>{Seminários individuais}
  \item<1-> \alert<6>{Seminários individuais}
  \item<1-> \alert<6>{Seminários individuais}
  \end{enumerate}
  \uncover<6>{ Os itens em vermelho indicam entregas de avaliação.}
\end{frame}

\begin{frame}{Avaliação}
  \begin{block}{Primeira entrega}
    Proposta inicial (sumário)
  \end{block}

  \pause

  \begin{block}{Segunda entrega}
    \begin{itemize}
    \item Projeto (padrão ABNT);
    \item Rascunho da apresentação da defesa.
    \end{itemize}
  \end{block}

  % \begin{itemize}
  % \item Escrita de um projeto {\em ad-hoc}.
  %       \end{itemize}
  % \item<3-> Seminário de defesa do projeto.
  % \end{itemize}
\end{frame}

\section{Ciência e Cientometria}

\subsection{Conceitos preliminares}

% \subsection{Verdade}

\begin{frame}{Verdade}
  \begin{columns}
    \begin{column}{6cm}<1-> ``{\em When you have eliminated the
        impossible, whatever remains, however improbable, must be the
        truth.}'' Sherlock Holmes
    \end{column}
    \begin{column}{4cm}<1->
      \includegraphics[height=0.4\textheight]{Intro/truth}
    \end{column}
  \end{columns}
\end{frame}

% \subsection{Evidências}

% \subsection{Dados, Informação e Conhecimento}

\begin{frame}{Dados, Informação e Conhecimento}
  % \item ``Evidence-based medicine (EBM) is the process of
  %   systematically reviewing, appraising and using clinical research
  %   findings to aid the delivery of optimum clinical care to
  %   patients''
    \begin{block}{Dados}
      elementos, códigos ou símbolos quantificáveis que são
    coletados em um experimento.
    \end{block}
    \begin{block}{Informação}
      agregação e interpretação de dados
    \end{block}
    \begin{block}{Conhecimento}
      agregação de um corpo de informações que tem significado e
      aplicabilidade prática
    \end{block}
\end{frame}


\begin{frame}{Dados, Informação e Conhecimento}
  \begin{center}
    \includegraphics[height=0.9\textheight]{Intro/Knowledge_pyramid}
  \end{center}
  % \begin{figure}
  %   \caption{}
  % \end{figure}
\end{frame}

% \begin{frame}{Dados, Informação e Conhecimento}
%   \begin{example}
%     Dados do mercado financeiro: preço das ações da Petrobrás hoje
%   \end{example}
% \end{frame}

% \subsection{Tipos de conhecimento}

% \begin{frame}{Conhecimento Filosófico}
%   \begin{itemize}
%   \item Valorativo (parte de hipóteses que não podem ser observadas)
%   \item Racional (enunciados logicamente correlacionados)
%   \item Sistemático
%   \item Não verificável (conclusões não podem ser confirmadas nem
%     refutadas)
%   \item Infalível e exato (razão pura)
%   \end{itemize}
% \end{frame}

% \begin{frame}{Conhecimento Teológico}
%   \begin{itemize}
%   \item Valorativo
%   \item Inspiracional
%   \item Sistemático
%   \item Não verificável
%   \item Infalível e exato (divino)
%   \end{itemize}
% \end{frame}

% \begin{frame}{Conhecimento Popular}
%   \begin{itemize}
%   \item Valorativo (estados de ânimo e emoções)
%   \item Reflexivo (não pode ser generalizado)
%   \item Assistemático
%   \item Verificável (no cotidiano)
%   \item Falível e inexato (percepções do dia a dia)
%   \end{itemize}
% \end{frame}

\begin{frame}{Conhecimento Científico}
  \begin{itemize}
  \item Factual
  \item Contingente (\alert{experimento} ao invés de razão pura)
  \item Sistemático 
  \item \alert{Verificável}
  \item Falível (não é definitivo)
  \item Aproximadamente exato (novos dados podem derrubar teorias
    anteriores)
  \end{itemize}
\end{frame}

% \begin{frame}{Classificação da Ciência}
%   \begin{center}
%     \includegraphics[width=\textwidth]{Intro/ciencias}
%   \end{center}
% \end{frame}

% \subsection{Resumo}

% \begin{frame}{Conhecimento Científico x Senso Comum}
%   \begin{itemize}
%   \item Um mesmo objeto ou fenômeno \alert{pode} ser observado por
%     qualquer tipo de conhecimento
%   \item diferença: forma de observação
%   \item Ciência
%     \begin{itemize}
%     \item reprodutível
%     \item acúmulo incremental
%     \item autoavaliação ou correção
%     \end{itemize}
%   \end{itemize}
% \end{frame}

\begin{frame}{Conhecimento Científico x Senso Comum}
  \begin{columns}
    \begin{column}{5cm}
      \begin{center}
        \includegraphics[width=0.55\textwidth]{Intro/the_difference1}
      \end{center}
    \end{column}
    \begin{column}{5cm}
      \begin{center}
        \begin{center}
          \includegraphics[width=\textwidth]{Intro/the_difference2}
        \end{center}
      \end{center}
    \end{column}
  \end{columns}

https://xkcd.com/242/
\end{frame}

\begin{frame}{Exercício}
Escreva um procedimento detalhado para pregar um prego em um sabão

  \begin{center}
    \includegraphics[width=0.8\textwidth]{Intro/pregomartelo}
  \end{center}
\end{frame}

\end{document}
