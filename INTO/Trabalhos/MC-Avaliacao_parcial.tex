\everymath{\displaystyle}
%\documentclass[pdftex,a4paper]{article}
\documentclass[a4paper]{article}
%%classes: article, report, book, proc, amsproc

%%%%%%%%%%%%%%%%%%%%%%%%
%% Misc
% para acertar os acentos
\usepackage[brazilian]{babel}
%\usepackage[portuguese]{babel}
% \usepackage[english]{babel}
% \usepackage[T1]{fontenc}
% \usepackage[latin1]{inputenc}
\usepackage[utf8]{inputenc}
\usepackage{indentfirst}
\usepackage{fullpage}
% \usepackage{graphicx} %See PDF section
\usepackage{multicol}
\setlength{\columnseprule}{0.5pt}
\setlength{\columnsep}{20pt}
%%%%%%%%%%%%%%%%%%%%%%%%
%%%%%%%%%%%%%%%%%%%%%%%%
%% PDF support

\usepackage[pdftex]{color,graphicx}
% %% Hyper-refs
\usepackage[pdftex]{hyperref} % for printing
% \usepackage[pdftex,bookmarks,colorlinks]{hyperref} % for screen

%% \newif\ifPDF
%% \ifx\pdfoutput\undefined\PDFfalse
%% \else\ifnum\pdfoutput > 0\PDFtrue
%%      \else\PDFfalse
%%      \fi
%% \fi

%% \ifPDF
%%   \usepackage[T1]{fontenc}
%%   \usepackage{aeguill}
%%   \usepackage[pdftex]{graphicx,color}
%%   \usepackage[pdftex]{hyperref}
%% \else
%%   \usepackage[T1]{fontenc}
%%   \usepackage[dvips]{graphicx}
%%   \usepackage[dvips]{hyperref}
%% \fi

%%%%%%%%%%%%%%%%%%%%%%%%


%%%%%%%%%%%%%%%%%%%%%%%%
%% Math
\usepackage{amsmath,amsfonts,amssymb}
% para usar R de Real do jeito que o povo gosta
\usepackage{amsfonts} % \mathbb
% para usar as letras frescas como L de Espaco das Transf Lineares
% \usepackage{mathrsfs} % \mathscr

% Oferecer seno e tangente em pt, com os comandos usuais.
\providecommand{\sin}{} \renewcommand{\sin}{\hspace{2pt}\mathrm{sen}}
\providecommand{\tan}{} \renewcommand{\tan}{\hspace{2pt}\mathrm{tg}}

% dt of integrals = \ud t
\newcommand{\ud}{\mathrm{\ d}}
%%%%%%%%%%%%%%%%%%%%%%%%

\title{Avaliação Parcial da disciplina de Metodologia Científica}
\date{}
\author{Docente: Felipe Figueiredo\\
  \url{prof.felipefigueiredo@gmail.com}\\
  \url{http://sites.google.com/site/proffelipefigueiredo}
}
\begin{document}
\maketitle
\abstract{Nesta avaliação parcial, os alunos deverão redigir um texto
  dissertativo que simule um projeto de pesquisa. Aqui estão gráficos
  que descrevem um conjunto de dados preliminar, como se obtidos em um
  projeto piloto. Caberá aos alunos (a) formular um problema de
  pesquisa baseado nesses dados preliminares, (b) indicar que outros
  tipos de dados deverão ser coletados para atender às questões
  levantadas, (c) justificar suas escolhas, preferencialmente com
  algumas referências bibliográficas.

  Obs: Os dados preliminares aqui contidos são altamente fictícios,
  portanto não há pergunta ou resposta correta nesta avaliação, apenas
  bons argumentos. Confronte-os com suas expectativas e intuição.}

\newpage

%%%%%%%%%%%%%%%%%%%%%%%%
%% Título e cabeçalho
%\noindent\parbox[c]{.15\textwidth}{\includegraphics[width=.15\textwidth]{logo}}\hfill
\parbox[c]{.825\textwidth}{\raggedright%
  \sffamily {\LARGE

Metodologia Científica

Avaliação Parcial

\par\bigskip}
{Prof: Felipe Figueiredo\par}
{\url{http://sites.google.com/site/proffelipefigueiredo}\par}
}

Versão: \verb|20160927|

%%%%%%%%%%%%%%%%%%%%%%%%


%%%%%%%%%%%%%%%%%%%%%%%%

\section{Objetivos}
O objetivo primário desta avaliação é proporcionar ao discente uma
primeira oportunidade de redigir um anteprojeto, seguindo normas
adequadas à pesquisa científica, e ao programa de Pós-Graduação. Para
tal, serão avaliados a clareza na exposição textual, e a exposição do
material, seguindo os padrões estabelecidos de formatação de página,
parágrafo, referências e demais elementos.

O objetivo secundário desta avaliação é estimular tanto a criatividade
acadêmica como a habilidade de formulação de problemas.

\section{O trabalho}

\subsection{Tema}

O tema é de livre escolha.

Escolha um tema que te motive, e que você esteja genuinamente interessado em resolver.
Pense em algo que você poderia trabalhar para sua dissertação.

Alternativamente, você pode propor a solução de um problema dentro de uma das linhas de pesquisa do programa de PG.

\subsection{Projeto}

Tomando como base o problema escolhido, você deve formular:

\begin{itemize}
\item um problema de pesquisa (pergunta)
\item hipóteses
\item um plano de execução da pesquisa
\end{itemize}

Este projeto deve ser redigido conforme as normas desta instituição
para o seminário de acompanhamento discente. A
descrição deste formato está na página da instituição. Quaisquer
dúvidas quanto ao formato podem ser tiradas com o docente.

Sugere-se que os grupos dividam a elaboração do projeto acompanhando
as aulas, redigindo um rascunho inicial de cada parte do documento
progressivamente, seguindo as recomendações oferecidas nas aulas. Este
rascunho inicial de cada parte (Introdução, Metodologia, Resultados,
\ldots) deve ser revisto periodicamente, para o refinamento da
qualidade textual.

Após a entrega do projeto textual, o grupo deverá apresentar um
seminário onde defenderá os argumentos apresentados no projeto. A
avaliação desta disciplina é composta tanto pelo projeto quanto pela
apresentação, com pesos iguais.

\subsection{Entrega e avaliação}

A avaliação será predominantemente quanto à forma, apresentação do projeto e aderência à norma ABNT.
Como de costume, seções excepcionais podem receber remuneração extra.
Dedique-se!

O projeto deve ser convertido para o formato PDF para entrega.
Após a conversão para PDF ele deve ser enviado por e-mail ( \url{prof.felipefigueiredo@gmail.com}), identificando no assunto do e-mail a instituição e nome da disciplina.
No {\bf corpo do e-mail} você deve identificar seu nome, e título do trabalho.
A correta identificação do trabalho é fundamental para o leitor, portanto isto também será considerado na pontuação do mesmo.

A entrega deve ocorrer no prazo agendado na página \url{https://sites.google.com/site/proffelipefigueiredo/into/metodologia-da-pesquisa-aplicada}.
Atrasos no envio do trabalho serão descontados proporcionalmente da pontuação do mesmo.
Um atraso de mais de 48h na entrega inviabiliza a avaliação.

Todos os PDFs serão escaneados num programa detector de plágio.
O plágio total ou parcial é um critério eliminatório nesta avaliação.

\section{Seções do projeto}

\subsection{Seções requeridas}

Cada seção (capítulo) do projeto deve iniciar em uma nova página. O projeto deve
conter minimamente os seguintes itens:

\begin{itemize}
\item Capa
\item Sumário
\item Resumo
\item Justificativa (sugestão 1 página)
\item Revisão da literatura (sugestão 1--2 páginas)
\item Objetivos (1 página)
\item Metodologia (sugestão 1 página)
\item Cronograma de atividades (sugestão 1 página)
\item Referências bibliográficas (sugestão 2--6 referências)
\end{itemize}

John Doe será o orientador deste projeto. Seu nome deverá ser
preenchido na capa do relatório de acompanhamento discente (disponível
no site da instituição).

\subsection{Cronograma}

O projeto deve incluir um cronograma, descrevendo metas para um plano
a ser executado em dois anos. O cronograma do projeto deve incluir:

\begin{itemize}
\item Revisão bibliográfica inicial (sugestão 2--6 meses)
\item Coleta de dados (sugestão 12--18 meses)
\item Análise de dados (sugestão 1--3 meses)
\item (opcional) Apresentação de resultados preliminares em
  seminário/congresso
\item Revisão bibliográfica complementar e discussão dos resultados
  (sugestão 2--6 meses)
\item Redação do relatório principal (sugestão 2--4 meses)
\item Redação de artigo para submissão em periódico científico
\end{itemize}

Os itens dos cronograma podem se sobrepor caso seja necessário, sempre
observando a viabilidade da sobreposição de tarefas\footnote{Observe
  que estas sugestões são para este projeto fictício, e não uma regra
  para projetos de Mestrado.}.

% \section{Problema}

% \begin{itemize}
% \item existe uma correlação entre a circunferência abdominal e doença
%   cardíaca?
% \end{itemize}

\end{document}
