\everymath{\displaystyle}
%\documentclass[pdftex,a4paper]{article}
\documentclass[a4paper]{article}
%%classes: article, report, book, proc, amsproc

%%%%%%%%%%%%%%%%%%%%%%%%
%% Misc
% para acertar os acentos
\usepackage[brazilian]{babel}
%\usepackage[portuguese]{babel}
% \usepackage[english]{babel}
% \usepackage[T1]{fontenc}
% \usepackage[latin1]{inputenc}
\usepackage[utf8]{inputenc}
\usepackage{indentfirst}
\usepackage{fullpage}
% \usepackage{graphicx} %See PDF section
\usepackage{multicol}
\setlength{\columnseprule}{0.5pt}
\setlength{\columnsep}{20pt}
%%%%%%%%%%%%%%%%%%%%%%%%
%%%%%%%%%%%%%%%%%%%%%%%%
%% PDF support

\usepackage[pdftex]{color,graphicx}
% %% Hyper-refs
\usepackage[pdftex]{hyperref} % for printing
% \usepackage[pdftex,bookmarks,colorlinks]{hyperref} % for screen

%% \newif\ifPDF
%% \ifx\pdfoutput\undefined\PDFfalse
%% \else\ifnum\pdfoutput > 0\PDFtrue
%%      \else\PDFfalse
%%      \fi
%% \fi

%% \ifPDF
%%   \usepackage[T1]{fontenc}
%%   \usepackage{aeguill}
%%   \usepackage[pdftex]{graphicx,color}
%%   \usepackage[pdftex]{hyperref}
%% \else
%%   \usepackage[T1]{fontenc}
%%   \usepackage[dvips]{graphicx}
%%   \usepackage[dvips]{hyperref}
%% \fi

%%%%%%%%%%%%%%%%%%%%%%%%


%%%%%%%%%%%%%%%%%%%%%%%%
%% Math
\usepackage{amsmath,amsfonts,amssymb}
% para usar R de Real do jeito que o povo gosta
\usepackage{amsfonts} % \mathbb
% para usar as letras frescas como L de Espaco das Transf Lineares
% \usepackage{mathrsfs} % \mathscr

% Oferecer seno e tangente em pt, com os comandos usuais.
\providecommand{\sin}{} \renewcommand{\sin}{\hspace{2pt}\mathrm{sen}}
\providecommand{\tan}{} \renewcommand{\tan}{\hspace{2pt}\mathrm{tg}}

% dt of integrals = \ud t
\newcommand{\ud}{\mathrm{\ d}}
%%%%%%%%%%%%%%%%%%%%%%%%

\title{Avaliação Parcial da disciplina de Metodologia Científica}
\date{}
\author{Prof. Felipe Figueiredo\\
\url{http://sites.google.com/site/proffelipefigueiredo}
}
\begin{document}
\maketitle
\newpage

%%%%%%%%%%%%%%%%%%%%%%%%
%% Título e cabeçalho
%\noindent\parbox[c]{.15\textwidth}{\includegraphics[width=.15\textwidth]{logo}}\hfill
\parbox[c]{.825\textwidth}{\raggedright%
  \sffamily {\LARGE

Metodologia Científica

Avaliação Parcial

\par\bigskip}
{Prof: Felipe Figueiredo\par}
{\url{http://sites.google.com/site/proffelipefigueiredo}\par}
}

Versão: \verb|YYYYMMDD|

%%%%%%%%%%%%%%%%%%%%%%%%


%%%%%%%%%%%%%%%%%%%%%%%%

\section{Objetivos}
O objetivo primário desta avaliação é proporcionar ao discente uma
primeira oportunidade de redigir um anteprojeto, seguindo normas
adequadas à pesquisa científica, e ao programa de Pós-Graduação. Para
tal, serão avaliados a clareza na exposição textual, e a exposição do
material, seguindo os padrões estabelecidos de formatação de página,
parágrafo, referências e demais elementos.

O objetivo secundário desta avaliação é estimular tanto a criatividade
acadêmica como a habilidade de formulação de problemas.


\section{Contexto}

\begin{itemize}
\item objetivo: os alunos devem, já sabendo a resposta por experiência
  própria, elaborar um projeto fictício para testar essa pergunta.
\item premissa: o dataset é um preliminar, de um estudo piloto, que
  servirá de base para o texto.
\item Devem pensar em como formular a hipótese, e descrever os
  resultados preliminares.
\end{itemize}
\section{Problema}

\begin{itemize}
\item existe uma correlação entre a circunferência abdominal e doença
  cardíaca?
\end{itemize}

\begin{figure}[h]
  \centering
  \includegraphics[width=.5\textwidth]{dispersao}
  \caption{Gráfico de dispersão da Circunferência abdominal (CA) de
    cada paciente.}
  \label{fig:dispersao}
\end{figure}

\begin{figure}[h]
  \centering
  \includegraphics[width=.5\textwidth]{histograma}
  \caption{Histograma de frequências da Circunferência abdominal (CA)
    de homens adultos.}
  \label{fig:histograma}
\end{figure}

\begin{figure}[h]
  \centering
  \includegraphics[width=.5\textwidth]{boxplot}
  \caption{Boxplot da Circunferência abdominal (CA) de homens
    adultos.}
  \label{fig:boxplot}
\end{figure}

\end{document}
