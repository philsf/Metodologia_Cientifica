\everymath{\displaystyle}
%\documentclass[pdftex,a4paper]{article}
\documentclass[a4paper]{article}
%%classes: article, report, book, proc, amsproc

%%%%%%%%%%%%%%%%%%%%%%%%
%% Misc
% para acertar os acentos
\usepackage[brazilian]{babel} 
%\usepackage[portuguese]{babel} 
% \usepackage[english]{babel}
% \usepackage[T1]{fontenc}
% \usepackage[latin1]{inputenc}
\usepackage[utf8]{inputenc}
\usepackage{indentfirst}
\usepackage{fullpage}
% \usepackage{graphicx} %See PDF section
\usepackage{multicol}
\setlength{\columnseprule}{0.5pt}
\setlength{\columnsep}{20pt}
%%%%%%%%%%%%%%%%%%%%%%%%
%%%%%%%%%%%%%%%%%%%%%%%%
%% PDF support

\usepackage[pdftex]{color,graphicx}
% %% Hyper-refs
\usepackage[pdftex]{hyperref} % for printing
% \usepackage[pdftex,bookmarks,colorlinks]{hyperref} % for screen

%% \newif\ifPDF
%% \ifx\pdfoutput\undefined\PDFfalse
%% \else\ifnum\pdfoutput > 0\PDFtrue
%%      \else\PDFfalse
%%      \fi
%% \fi

%% \ifPDF
%%   \usepackage[T1]{fontenc}
%%   \usepackage{aeguill}
%%   \usepackage[pdftex]{graphicx,color}
%%   \usepackage[pdftex]{hyperref}
%% \else
%%   \usepackage[T1]{fontenc}
%%   \usepackage[dvips]{graphicx}
%%   \usepackage[dvips]{hyperref}
%% \fi

%%%%%%%%%%%%%%%%%%%%%%%%


%%%%%%%%%%%%%%%%%%%%%%%%
%% Math
\usepackage{amsmath,amsfonts,amssymb}
% para usar R de Real do jeito que o povo gosta
\usepackage{amsfonts} % \mathbb
% para usar as letras frescas como L de Espaco das Transf Lineares
% \usepackage{mathrsfs} % \mathscr

% Oferecer seno e tangente em pt, com os comandos usuais.
\providecommand{\sin}{} \renewcommand{\sin}{\hspace{2pt}\mathrm{sen}}
\providecommand{\tan}{} \renewcommand{\tan}{\hspace{2pt}\mathrm{tg}}

% dt of integrals = \ud t
\newcommand{\ud}{\mathrm{\ d}}
%%%%%%%%%%%%%%%%%%%%%%%%
\date{
\bigskip
Curso: \underline{\hspace{8cm}}\\
\ \\
Turma: \underline{\hspace{1cm}} Série: \underline{\hspace{1cm}} Turno:
\underline{\hspace{1cm}}\\
\ \\
Prof: \underline{\hspace{8cm}}\\
}

\title{Metodologia Científica - Trabalho 1}

\author{
{\bf Grupo} \underline{\hspace{1cm}}\\
\ \\
Nome: \underline{\hspace{6cm}} RA: \underline{\hspace{2cm}} Assinatura: \underline{\hspace{4cm}}\\
Nome: \underline{\hspace{6cm}} RA: \underline{\hspace{2cm}} Assinatura: \underline{\hspace{4cm}}\\
Nome: \underline{\hspace{6cm}} RA: \underline{\hspace{2cm}} Assinatura: \underline{\hspace{4cm}}\\
Nome: \underline{\hspace{6cm}} RA: \underline{\hspace{2cm}} Assinatura: \underline{\hspace{4cm}}\\
Nome: \underline{\hspace{6cm}} RA: \underline{\hspace{2cm}} Assinatura: \underline{\hspace{4cm}}\\
Nome: \underline{\hspace{6cm}} RA: \underline{\hspace{2cm}} Assinatura: \underline{\hspace{4cm}}\\
Nome: \underline{\hspace{6cm}} RA: \underline{\hspace{2cm}} Assinatura: \underline{\hspace{4cm}}\\
Nome: \underline{\hspace{6cm}} RA: \underline{\hspace{2cm}} Assinatura: \underline{\hspace{4cm}}\\
Nome: \underline{\hspace{6cm}} RA: \underline{\hspace{2cm}} Assinatura: \underline{\hspace{4cm}}\\
Nome: \underline{\hspace{6cm}} RA: \underline{\hspace{2cm}} Assinatura: \underline{\hspace{4cm}}\\
}

% Usar tabela do RStudio
\usepackage{longtable,booktabs}

\begin{document}
%\maketitle
\newpage

%%%%%%%%%%%%%%%%%%%%%%%%
%% Título e cabeçalho
%\noindent\parbox[c]{.15\textwidth}{\includegraphics[width=.15\textwidth]{logo}}\hfill
\parbox[c]{.825\textwidth}{\raggedright%
  \sffamily {\LARGE

Metodologia Científica: Trabalho 1

\par\bigskip}
{Prof: Felipe Figueiredo\par}
{\url{http://sites.google.com/site/proffelipefigueiredo}\par}
}

Versão: \verb|20160928|

%%%%%%%%%%%%%%%%%%%%%%%%


%%%%%%%%%%%%%%%%%%%%%%%%


\section{Orientações gerais}

\begin{itemize}
\item A turma deve se dividir em 8 grupos, de até 10 alunos.
\item O texto do Trabalho 1 deverá ser enviado para o e-mail da turma de modo a que todos tenham acesso, e com cópia para o professor (\url{prof.felipefigueiredo@gmail.com}).
\item Além da parte escrita, o grupo deve fazer uma apresentação para toda a turma, explicando o que foi estudado e tirando as dúvidas dos colegas.
\item Cada grupo terá até 20 minutos para a apresentação e 5 minutos para questionamentos.
\item Entrega da parte escrita: e-mails até o dia 23/03.
\item Cronograma de apresentação: Grupos I a V no dia 24/03, e grupos VI a IX no dia 31/03.
\end{itemize}

\section{Valor}
Este trabalho vale até $3.0$ pontos que serão somados com a nota da P1, compondo assim a nota do primeiro bimestre (B1).

\begin{itemize}
\item 1,0 parte escrita
\item 1,0 montagem da apresentação
\item 1,0 participação do aluno na apresentação
\end{itemize}


% entregue na aula e horário normais da semana indicada no endereço abaixo:

% \url{https://sites.google.com/site/proffelipefigueiredo/anhanguera/2016}.

\section{Sugestões}

\begin{itemize}
\item Além do livro referenciado aqui, pesquise sobre os temas de seu trabalho em livros e/ou na internet. Não faça cópia pois isso caracteriza plágio e poderá zerar seu trabalho.
\item Anote todas as fontes de consulta para fazer a referência bibliográfica.
\end{itemize}



\section{Formatação}

\begin{itemize}
\item A capa deverá conter as seguintes informações: instituição, curso, disciplina, professor, tema e data.
\item A contracapa deverá conter o nome e RA dos membros do grupo em ordem alfabética.
\item Corpo do trabalho:

Margens: superior e esquerda 3cm, inferior e direita 2,0 cm.

Fonte: Arial, 12, cor preta.

Espaço entre linha: 1,5

Título: centralizado, negrito, letras maiúsculas, 16.

Subtítulo: negrito, letras maiúsculas, 14.

Bibliografia: em ordem alfabética. Para citar sítio da internet, adicionar data da consulta.

\end{itemize}





\newpage
\section{Temas}

Os tópicos a seguir compõem boa parte do capítulo 3 do livro {\em Metodologia do Trabalho Científico}, de Cléber Prodanov e Ernâni Freitas (2013, 2a edição).

A referência completa, bem como todas as informações da disciplina estão no endereço

\url{https://sites.google.com/site/proffelipefigueiredo/anhanguera/disciplinas/metodologia-cientifica}

Além dessa fonte, o grupo pode buscar outras fontes de livre escolha para complementar a discussão, como outros livros, sítios e vídeos.



% \begin{enumerate}
% \item 2.1, 2.2
% \item 2.2 e 2.3
% \item 2.3 e 2.4.1.1 e 2.4.1.2
% \item 2.4.1.1, 2.4.1.2 e 2.4.1.3
% \item 2.4.1.2 e 2.4.1.3
% \item 2.4.1.4 e 2.1.4.5
% \item 2.4.2.2 e 2.4.3.3
% \item 2.5
% \end{enumerate}

\begin{multicols}{2}
{\bf Grupo I}

\begin{itemize}
\item 3.1 O que é pesquisa? (pg 42)
\item 3.3 Características da Pesquisa Científica (pg 48)
\item 3.4 Classificação das Pesquisas (pg 49)
\item 3.4.1 Do ponto de vista natureza (pg 51)
\item 3.4.2 Do ponto de vista  objetivos (pg 51)
\end{itemize}

{\bf Grupo II}
\begin{itemize}
\item 3.4.3  ponto de vista  procedimentos (pg 54)
\end{itemize}

{\bf Grupo III}
\begin{itemize}
\item 3.4.4 Do ponto de vista  abordagem (pg 69)
\end{itemize}

{\bf Grupo IV}
\begin{itemize}
\item 3.5 Etapas da Pesquisa (pg 73)
\item 3.5.1 O planejamento da pesquisa (pg 73)
\item 3.5.2 Atitudes do pesquisador (pg 73)
\item 3.5.3 Fases da pesquisa (pg 74)
\item a Formulação planejamento  (pg 75)
\item b escolha assunto, delimitação tema (pg 75)
\end{itemize}

{\bf Grupo V}

\begin{itemize}
\item c revisão literatura (pg 78)
\item d justificativa (pg 82)
\end{itemize}

\columnbreak

{\bf Grupo VI}
\begin{itemize}
\item e problema (pg 83)
\item f hipóteses (pg 88)
\end{itemize}

{\bf Grupo VII}
\begin{itemize}
\item g objetivos: geral e específicos (pg 94)
\item h coleta dados (pg 97)
\end{itemize}

{\bf Grupo VIII}
\begin{itemize}
\item técnicas de pesquisa coleta (pg 102)
\item questionários (pg 105)
\end{itemize}

{\bf Grupo XI}
\begin{itemize}
\item i tabulação dados (pg 112)
\item j analise interpretação dados (pg 112)
\item k conclusão (pg 116)
\item l redação (pg 117)
\end{itemize}

\end{multicols}

\end{document}
