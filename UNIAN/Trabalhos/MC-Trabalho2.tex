\everymath{\displaystyle}
%\documentclass[pdftex,a4paper]{article}
\documentclass[a4paper]{article}
%%classes: article, report, book, proc, amsproc

%%%%%%%%%%%%%%%%%%%%%%%%
%% Misc
% para acertar os acentos
\usepackage[brazilian]{babel}
%\usepackage[portuguese]{babel}
% \usepackage[english]{babel}
% \usepackage[T1]{fontenc}
% \usepackage[latin1]{inputenc}
\usepackage[utf8]{inputenc}
\usepackage{indentfirst}
\usepackage{fullpage}
% \usepackage{graphicx} %See PDF section
\usepackage{multicol}
\setlength{\columnseprule}{0.5pt}
\setlength{\columnsep}{20pt}
%%%%%%%%%%%%%%%%%%%%%%%%
%%%%%%%%%%%%%%%%%%%%%%%%
%% PDF support

\usepackage[pdftex]{color,graphicx}
% %% Hyper-refs
\usepackage[pdftex]{hyperref} % for printing
% \usepackage[pdftex,bookmarks,colorlinks]{hyperref} % for screen

%% \newif\ifPDF
%% \ifx\pdfoutput\undefined\PDFfalse
%% \else\ifnum\pdfoutput > 0\PDFtrue
%%      \else\PDFfalse
%%      \fi
%% \fi

%% \ifPDF
%%   \usepackage[T1]{fontenc}
%%   \usepackage{aeguill}
%%   \usepackage[pdftex]{graphicx,color}
%%   \usepackage[pdftex]{hyperref}
%% \else
%%   \usepackage[T1]{fontenc}
%%   \usepackage[dvips]{graphicx}
%%   \usepackage[dvips]{hyperref}
%% \fi

%%%%%%%%%%%%%%%%%%%%%%%%


%%%%%%%%%%%%%%%%%%%%%%%%
%% Math
\usepackage{amsmath,amsfonts,amssymb}
% para usar R de Real do jeito que o povo gosta
\usepackage{amsfonts} % \mathbb
% para usar as letras frescas como L de Espaco das Transf Lineares
% \usepackage{mathrsfs} % \mathscr

% Oferecer seno e tangente em pt, com os comandos usuais.
\providecommand{\sin}{} \renewcommand{\sin}{\hspace{2pt}\mathrm{sen}}
\providecommand{\tan}{} \renewcommand{\tan}{\hspace{2pt}\mathrm{tg}}

% dt of integrals = \ud t
\newcommand{\ud}{\mathrm{\ d}}
%%%%%%%%%%%%%%%%%%%%%%%%

\title{Metodologia Científica: Trabalho 2}

\date{}
\author{Docente: Felipe Figueiredo\\
  \url{prof.felipefigueiredo@gmail.com}\\
  \url{http://sites.google.com/site/proffelipefigueiredo}
}
\begin{document}
\maketitle
\abstract{
Nesta avaliação o discente deverá redigir um texto dissertativo na forma de projeto de pesquisa.
Para isto você deve (a) formular um problema de pesquisa, (b) indicar que tipos de dados deverão ser coletados para responder às questões levantadas, (c) justificar suas escolhas.

Obs: não há pergunta ou resposta correta nesta avaliação, apenas bons argumentos.
Use suas expectativas e intuição.
}

\newpage

%%%%%%%%%%%%%%%%%%%%%%%%
%% Título e cabeçalho
%\noindent\parbox[c]{.15\textwidth}{\includegraphics[width=.15\textwidth]{logo}}\hfill
\parbox[c]{.825\textwidth}{\raggedright%
  \sffamily {\LARGE

Metodologia Científica: Trabalho 2

\par\bigskip}
{Prof: Felipe Figueiredo\par}
{\url{http://sites.google.com/site/proffelipefigueiredo}\par}
}

Versão: \verb|20160505|

%%%%%%%%%%%%%%%%%%%%%%%%


%%%%%%%%%%%%%%%%%%%%%%%%

\section{Objetivos}
O objetivo primário desta avaliação é proporcionar ao discente uma primeira oportunidade de redigir um anteprojeto, seguindo normas adequadas à pesquisa científica, e à Instituição de Ensino Superior no padrão ABNT.
Para tal, serão avaliados a clareza na exposição textual, e a exposição do material, seguindo os padrões estabelecidos de formatação de página, parágrafo, referências e demais elementos.

O objetivo secundário desta avaliação é estimular tanto a criatividade acadêmica como a habilidade de formulação de problemas.

\section{O trabalho}

\subsection{Tema}

O tema é de livre escolha.

Escolha um tema que te motive, e que você esteja genuinamente interessado em resolver.
Pense em algo que você poderia trabalhar para seu TCC.

Alternativamente, você pode propor a solução de um problema da sua vivência cotidiana, não é necessário que esteja dentro das disciplinas que já cursou.

\subsection{Projeto}

Tomando como base o problema escolhido, você deve formular:

\begin{itemize}
\item um problema de pesquisa (pergunta)
\item hipóteses
\item um plano de execução da pesquisa
\end{itemize}

Este projeto deve ser formatado conforme as normas ABNT para trabalhos acadêmicos.
A descrição deste formato está na página da instituição, e será esmiuçada ao longo das aulas.
Quaisquer dúvidas quanto ao formato podem ser tiradas com o docente presencialmente ou por e-mail, no endereço \url{prof.felipefigueiredo@gmail.com}.
Diversas informações úteis estão listadas no endereço \url{https://sites.google.com/site/proffelipefigueiredo/anhanguera/disciplinas/metodologia-cientifica}.

\subsection{Sugestões}

Comece fazendo uma breve redação (de 1 a 2 páginas) sobre o tema que você escolheu.
Explique o contexto do problema escolhido e justifique a importância da resolução do mesmo.
Esse esforço inicial será posteriormente aproveitado nas várias seções do seu projeto.

Busque fontes de consulta sobre o assunto, e anote as referências para citação posterior.

Se você fizer um pouco a cada semana, não vai ficar muita coisa para o final.
Divida a elaboração do projeto acompanhando as aulas, fazendo um rascunho inicial de cada parte do documento progressivamente, seguindo as recomendações oferecidas nas aulas.
Este rascunho inicial de cada parte (Introdução, Objetivos, Metodologia, \ldots) deve ser revisto periodicamente, para o refinamento da qualidade textual.

É fácil encontrar na internet modelos de documento pré-formatado, que podem auxiliar a parte ``burocrática'' do trabalho, te livrando para pensar e escrever sobre seu problema.

\subsection{Entrega e avaliação}

O projeto deve ser convertido para o formato PDF para entrega.

Após a conversão para PDF ele deve ser enviado por e-mail ( \url{prof.felipefigueiredo@gmail.com}), identificando no assunto do e-mail a turma e disciplina do trabalho.
No {\bf corpo do e-mail} você deve identificar seu nome, RA e título do trabalho.
A correta identificação do trabalho é fundamental para o leitor, portanto isto também será considerado na pontuação do mesmo.
A entrega deve ocorrer até 23:59 da data agendada na página \url{https://sites.google.com/site/proffelipefigueiredo/anhanguera/2016-1}.

A avaliação será predominantemente quanto à forma, apresentação do projeto e aderência à norma ABNT.
Como de costume, seções excepcionais podem receber remuneração extra.
Dedique-se!

Todos os PDFs serão escaneados num programa detector de plágio.
O plágio total ou parcial é um critério eliminatório nesta avaliação.
Atrasos no envio do trabalho serão descontados proporcionalmente da pontuação do mesmo.
Um atraso de mais de 48h na entrega inviabiliza a avaliação.

\section{Capítulos do projeto}

\subsection{Itens requeridos}

Cada capítulo do projeto deve iniciar em uma nova página.
O projeto deve conter obrigatoriamente os seguintes itens e capítulos:

\begin{itemize}
\item Capa (opcional: contracapa)
\item Resumo
\item Sumário
\item Justificativa (sugestão 2--3 parágrafos)
\item Introdução (sugestão 1--2 páginas)
\item Objetivos
\item Metodologia (sugestão 1 página)
\item Cronograma de atividades (sugestão 1 página)
\item Referências bibliográficas
\end{itemize}

\subsection{Cronograma}

O projeto deve incluir um cronograma, descrevendo metas para um plano
a ser executado em um ano.
O cronograma do projeto deve incluir:

\begin{itemize}
\item Revisão bibliográfica (sugestão 2--6 meses)
\item Coleta de dados (sugestão 2--6 meses)
\item Análise de dados (sugestão 1--2 meses)
\item Redação do relatório técnico ao final do projeto (sugestão 1--2 meses)
\item (opcional) Apresentação de resultados preliminares em seminário/congresso/encontro empresarial ou industrial
\item (opcional) Redação de artigo para submissão em periódico científico
\end{itemize}

Os itens dos cronograma podem se sobrepor caso seja necessário, sempre
observando a viabilidade da sobreposição de tarefas\footnote{Observe
  que estas sugestões são para este projeto fictício, e não uma regra
  para projetos de TCC.}.



% \section{Problema}

% \begin{itemize}
% \item existe uma correlação entre a circunferência abdominal e doença
%   cardíaca?
% \end{itemize}

\end{document}
