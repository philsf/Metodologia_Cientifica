\everymath{\displaystyle}
\documentclass{beamer}
% \documentclass[handout]{beamer}

%\usepackage[pdftex]{color,graphicx}
\usepackage{amsmath,amssymb,amsfonts}

\mode<presentation>
{
  % \usetheme{Darmstadt}
  % \usetheme[hideothersubsections]{Hannover}
  % \usetheme[hideothersubsections]{Goettingen}
  \usetheme[hideothersubsections, right]{Berkeley}

  \usecolortheme{seahorse}
  % \usecolortheme{dolphin}
  \usecolortheme{rose}
  % \usecolortheme{orchid}

  \useinnertheme[shadow]{rounded}

  \setbeamercovered{transparent}
  % or whatever (possibly just delete it)
}

\mode<handout>{
  \setbeamercolor{background canvas}{bg=black!5}
  \usepackage{pgfpages}
  \pgfpagesuselayout{4 on 1}[a4paper,border shrink=5mm, landscape]
}

\usepackage[brazilian]{babel}
% or whatever

% \usepackage[latin1]{inputenc}
\usepackage[utf8]{inputenc}
% or whatever

\usepackage{times}
%\usepackage[T1]{fontenc}
% Or whatever. Note that the encoding and the font should match. If T1
% does not look nice, try deleting the line with the fontenc.


\title%[] % (optional, use only with long paper titles)
{Indicadores em Ciência}

\subtitle
{Fator de Impacto, Índice H, et al} % (optional)

\author%[] % (optional, use only with lots of authors)
{Felipe Figueiredo}% \and S.~Another\inst{2}}
% - Use the \inst{?} command only if the authors have different
%   affiliation.

\institute[INTO] % (optional, but mostly needed)
{Instituto Nacional de Traumatologia e Ortopedia
}
  % \inst{1}%
  % Department of Computer Science\\
  % University of Somewhere
  % \and
  % \inst{2}%
  % Department of Theoretical Philosophy\\
  % University of Elsewhere}
% - Use the \inst command only if there are several affiliations.
% - Keep it simple, no one is interested in your street address.

\date%[] % (optional)
{}

% \subject{Talks}
% This is only inserted into the PDF information catalog. Can be left
% out. 



% If you have a file called "university-logo-filename.xxx", where xxx
% is a graphic format that can be processed by latex or pdflatex,
% resp., then you can add a logo as follows:

\pgfdeclareimage[height=1.6cm]{university-logo}{../logo}
\logo{\pgfuseimage{university-logo}}



% Delete this, if you do not want the table of contents to pop up at
% the beginning of each subsection:
\AtBeginSubsection[]
%\AtBeginSection[]
{
  \begin{frame}<beamer>{Sumário}
    \tableofcontents[currentsection,currentsubsection]
  \end{frame}
}


% If you wish to uncover everything in a step-wise fashion, uncomment
% the following command: 

\beamerdefaultoverlayspecification{<+->}


\begin{document}

\begin{frame}
  \titlepage
\end{frame}

\begin{frame}{Sumário}
  \tableofcontents
  % You might wish to add the option [pausesections]
\end{frame}


%% Template
% \section{}

% \subsection{}

% \begin{frame}{}
%   \begin{itemize}
%   \item 
%   \end{itemize}
% \end{frame}

% \begin{frame}
%   \begin{columns}
%     \begin{column}{5cm}
%     \end{column}
%     \begin{column}{5cm}
%     \end{column}
%   \end{columns}
% \end{frame}

% \begin{frame}{}
%   \includegraphics[height=0.4\textheight]{file1}
%   \includegraphics[height=0.4\textheight]{file2}
%   \includegraphics[height=0.4\textheight]{file3}
%   \begin{figure}
%     \caption{}
%   \end{figure}
% \end{frame}

% \begin{frame}{}
%   \begin{definition}
%   \end{definition}
%   \begin{example}
%   \end{example}
%   \begin{block}{Exercício}
%   \end{block}
% \end{frame}

\section{Indicadores de Revistas}

\subsection{Relevância}

\subsection{Fator de Impacto}

\subsection{Qualis da CAPES}

\section{Indicadores de Pesquisadores}

\subsection{Algumas propostas}

\begin{frame}{Como {\em medir} a ``relevância'' de um pesquisador?}
  \begin{itemize}
  \item Como atribuir uma métrica objetiva à produção de um cientista?
  \item Como detectar trabalhos {\em revolucionários}?
  \item Como fazer tudo isso, respeitando nossa intuição (e.g. Newton, Einstein, Darwin, \ldots)?
  \end{itemize}
\end{frame}

\begin{frame}{Algumas propostas}
  \begin{itemize}
  \item Número de artigos publicados (total, ou por ano)
  \item Total de citações recebidas
  \item Número de citações por artigo
  \end{itemize}
\end{frame}

\begin{frame}{Número de artigos}
  \begin{block}{Premissa}
    Quanto maior a produtividade, maior a relevância do cientista.
  \end{block}
  \begin{itemize}
  \item ``Pastel chinês''
  \item Alguns autores produzem MUITOS artigos, incluindo muitos de qualidade
  \item Estes são exceção, não a regra
  \item Em geral, muitos artigos não implicam em muito conhecimento ou informação gerados
  \item A publicação só tem impacto, se é lida e usada como base para novos trabalhos (i.e.: citada)
  \end{itemize}
\end{frame}

\begin{frame}{Total de citações recebidas}
  \begin{block}{Premissa}
    Quanto mais citações recebidas, maior a relevância da produção para a comunidade.
  \end{block}
  \begin{itemize}
  \item Trabalhos muito citados inflacionam esta métrica
  \item Um pesquisador pode ter apenas um trabalho muito citado, e vários menos relevantes
  \item Pesquisadores mais antigos acumulam citações há mais tempo que os jovens
  \end{itemize}
\end{frame}

\begin{frame}{Total de citações por artigo}
  \begin{block}{Premissa}
    Média alta de citações por artigo indica uma produtividade média relevante
  \end{block}
  \begin{itemize}
  \item A média é ``melhor'' que o total, simplifica a análise (sumariza)
  \item Publicar muitos artigos, aumenta a dificuldade de manter uma média alta!
  \item Trabalhos muito citados também podem inflacionar a média (perda de relevância)
  \item Com o tempo, as citações também se acumulam nos trabalhos antigos
  \end{itemize}
\end{frame}

\subsection{Índice H}

\subsection{Índice M}

\section{Referências}

\begin{frame}{Referências}
  \begin{enumerate}
  \item<1-> HIRSCH, J.E. (2005) An index to quantify an individual's scientific research output, PNAS.
  \item<1-> GARFIELD, E. (2006) The History and Meaning of the Journal Impact Factor, JAMA.
  \item<1-> HUBBARD, S. C.; McVEIGH, M. E. (2011). Casting a wide net: The Journal Impact Factor numerator, Learned Publishing
  \item<1-> HIRSCH, J.E. (2007) Does the h index have predictive power?, PNAS.
  \end{enumerate}
\end{frame}

\end{document}
