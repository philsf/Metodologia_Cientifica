\everymath{\displaystyle}
\documentclass{beamer}
% \documentclass[handout]{beamer}

%\usepackage[pdftex]{color,graphicx}
\usepackage{amsmath,amssymb,amsfonts}

\mode<presentation>
{
  % \usetheme{Darmstadt}
  % \usetheme[hideothersubsections]{Hannover}
  % \usetheme[hideothersubsections]{Goettingen}
  \usetheme[hideothersubsections, right]{Berkeley}

  \usecolortheme{seahorse}
  % \usecolortheme{dolphin}
  \usecolortheme{rose}
  % \usecolortheme{orchid}

  \useinnertheme[shadow]{rounded}

  \setbeamercovered{transparent}
  % or whatever (possibly just delete it)
}

\mode<handout>{
  \setbeamercolor{background canvas}{bg=black!5}
  \usepackage{pgfpages}
  \pgfpagesuselayout{4 on 1}[a4paper,border shrink=5mm, landscape]
}

\usepackage[brazilian]{babel}
% or whatever

% \usepackage[latin1]{inputenc}
\usepackage[utf8]{inputenc}
% or whatever

\usepackage{times}
%\usepackage[T1]{fontenc}
% Or whatever. Note that the encoding and the font should match. If T1
% does not look nice, try deleting the line with the fontenc.


\title%[] % (optional, use only with long paper titles)
{Análise Exploratória de Dados}

\subtitle
{Formulação de perguntas de dados preliminares} % (optional)

\author%[] % (optional, use only with lots of authors)
{Felipe Figueiredo}% \and S.~Another\inst{2}}
% - Use the \inst{?} command only if the authors have different
%   affiliation.

\institute[INTO] % (optional, but mostly needed)
{Instituto Nacional de Traumatologia e Ortopedia
}
  % \inst{1}%
  % Department of Computer Science\\
  % University of Somewhere
  % \and
  % \inst{2}%
  % Department of Theoretical Philosophy\\
  % University of Elsewhere}
% - Use the \inst command only if there are several affiliations.
% - Keep it simple, no one is interested in your street address.

\date%[] % (optional)
{}

% \subject{Talks}
% This is only inserted into the PDF information catalog. Can be left
% out. 



% If you have a file called "university-logo-filename.xxx", where xxx
% is a graphic format that can be processed by latex or pdflatex,
% resp., then you can add a logo as follows:

\pgfdeclareimage[height=1.6cm]{university-logo}{../logo}
\logo{\pgfuseimage{university-logo}}



% Delete this, if you do not want the table of contents to pop up at
% the beginning of each subsection:
\AtBeginSubsection[]
%\AtBeginSection[]
{
  \begin{frame}<beamer>{Sumário}
    \tableofcontents[currentsection,currentsubsection]
  \end{frame}
}


% If you wish to uncover everything in a step-wise fashion, uncomment
% the following command: 

\beamerdefaultoverlayspecification{<+->}


\begin{document}

\begin{frame}
  \titlepage
\end{frame}

\begin{frame}{Sumário}
  \tableofcontents
  % You might wish to add the option [pausesections]
\end{frame}


%% Template
% \section{}

% \subsection{}

% \begin{frame}{}
%   \begin{itemize}
%   \item 
%   \end{itemize}
% \end{frame}

% \begin{frame}
%   \begin{columns}
%     \begin{column}{5cm}
%     \end{column}
%     \begin{column}{5cm}
%     \end{column}
%   \end{columns}
% \end{frame}

% \begin{frame}{}
%   \includegraphics[height=0.4\textheight]{file1}
%   \includegraphics[height=0.4\textheight]{file2}
%   \includegraphics[height=0.4\textheight]{file3}
%   \begin{figure}
%     \caption{}
%   \end{figure}
% \end{frame}

% \begin{frame}{}
%   \begin{definition}
%   \end{definition}
%   \begin{example}
%   \end{example}
%   \begin{block}{Exercício}
%   \end{block}
% \end{frame}

\section{Análise Exploratória}

\subsection{EDA}

\begin{frame}{Paradigmas de Análises de Dados}
  \begin{itemize}
  \item EDA -- Análise Exploratória de Dados
  \item CDA -- Análise Confirmatória de Dados
  \end{itemize}
\end{frame}

\begin{frame}{Análise Exploratória de Dados}
  \begin{itemize}
  \item Formalizado por John W. Tukey nos anos 1970
  \item Objetivo: formular perguntas com base nos dados disponíveis
  \item Perguntas que podem ser respondidas pela análise dos dados
  \end{itemize}
\end{frame}

\begin{frame}{Análise Exploratória de Dados}
  \begin{block}{Tukey, 1980}
    {\em ``Ideas come from previous exploration more often than from
      lightning strokes. Important questions can demand the most
      careful planning for confirmatory analysis. (\ldots) Finding the
      question is often more important than finding the
      answer. Exploratory data analysis is an atitude, (\ldots) NOT a
      bundle of techniques (\ldots).''}
  \end{block}

Tukey (1980) - We Need Both Exploratory and Confirmatory
\end{frame}

\begin{frame}{Paradigma linear}
  \begin{center}
    \includegraphics[width=0.7\textwidth]{tukey1}
  \end{center}
  \begin{enumerate}
  \item Como as perguntas são geradas?
  \item Como os desenhos (experimentais) são guiados?
  \item Como a coleta de dados é monitorada?
  \end{enumerate}
\end{frame}

\begin{frame}{Questões sobre esse paradigma linear}
  Respostas: Geralmente por \ldots
  \begin{enumerate}
  \item  {\em insights} teóricos e a exploração de dados
    anteriores (e.g., pesquisa bibliográfica)
  \item informação qualitativa disponível obtida da exploração de
    dados anteriores
  \item exploração dos dados, conforme são obtidos, buscando
    comportamento ``inesperado''
  \end{enumerate}
\end{frame}

\begin{frame}{Paradigma alternativo}
  \begin{center}
    \includegraphics[width=0.7\textwidth]{tukey2}
  \end{center}
  \begin{itemize}
  \item 
  \end{itemize}
\end{frame}


\subsection{Dados}

\subsection{Histogramas}

\subsection{Boxplot}


\end{document}
