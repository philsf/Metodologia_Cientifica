\everymath{\displaystyle}
\documentclass{beamer}
% \documentclass[handout]{beamer}

%\usepackage[pdftex]{color,graphicx}
\usepackage{amsmath,amssymb,amsfonts}

\mode<presentation>
{
  % \usetheme{Darmstadt}
  % \usetheme[hideothersubsections]{Hannover}
  % \usetheme[hideothersubsections]{Goettingen}
  \usetheme[hideothersubsections, right]{Berkeley}

  \usecolortheme{seahorse}
  % \usecolortheme{dolphin}
  \usecolortheme{rose}
  % \usecolortheme{orchid}

  \useinnertheme[shadow]{rounded}

  \setbeamercovered{transparent}
  % or whatever (possibly just delete it)
}

\mode<handout>{
  \setbeamercolor{background canvas}{bg=black!5}
  \usepackage{pgfpages}
  \pgfpagesuselayout{4 on 1}[a4paper,border shrink=5mm, landscape]
}

\usepackage[brazilian]{babel}
% or whatever

% \usepackage[latin1]{inputenc}
\usepackage[utf8]{inputenc}
% or whatever

\usepackage{times}
%\usepackage[T1]{fontenc}
% Or whatever. Note that the encoding and the font should match. If T1
% does not look nice, try deleting the line with the fontenc.


\title%[] % (optional, use only with long paper titles)
{Tópicos de Escrita Científica}

\subtitle
{} % (optional)

\author%[] % (optional, use only with lots of authors)
{Felipe Figueiredo}% \and S.~Another\inst{2}}
% - Use the \inst{?} command only if the authors have different
%   affiliation.

\institute[INTO] % (optional, but mostly needed)
{Instituto Nacional de Traumatologia e Ortopedia
}
  % \inst{1}%
  % Department of Computer Science\\
  % University of Somewhere
  % \and
  % \inst{2}%
  % Department of Theoretical Philosophy\\
  % University of Elsewhere}
% - Use the \inst command only if there are several affiliations.
% - Keep it simple, no one is interested in your street address.

\date%[] % (optional)
{}

% \subject{Talks}
% This is only inserted into the PDF information catalog. Can be left
% out. 



% If you have a file called "university-logo-filename.xxx", where xxx
% is a graphic format that can be processed by latex or pdflatex,
% resp., then you can add a logo as follows:

\pgfdeclareimage[height=1.6cm]{university-logo}{../logo}
\logo{\pgfuseimage{university-logo}}



% Delete this, if you do not want the table of contents to pop up at
% the beginning of each subsection:
\AtBeginSubsection[]
%\AtBeginSection[]
{
  \begin{frame}<beamer>{Sumário}
    \tableofcontents[currentsection,currentsubsection]
  \end{frame}
}


% If you wish to uncover everything in a step-wise fashion, uncomment
% the following command: 

\beamerdefaultoverlayspecification{<+->}


\begin{document}

\begin{frame}
  \titlepage
\end{frame}

\begin{frame}{Sumário}
  \tableofcontents
  % You might wish to add the option [pausesections]
\end{frame}


%% Template
% \section{}

% \subsection{}

% \begin{frame}{}
%   \begin{itemize}
%   \item 
%   \end{itemize}
% \end{frame}

% \begin{frame}
%   \begin{columns}
%     \begin{column}{5cm}
%     \end{column}
%     \begin{column}{5cm}
%     \end{column}
%   \end{columns}
% \end{frame}

% \begin{frame}{}
%   \includegraphics[height=0.4\textheight]{file1}
%   \includegraphics[height=0.4\textheight]{file2}
%   \includegraphics[height=0.4\textheight]{file3}
%   \begin{figure}
%     \caption{}
%   \end{figure}
% \end{frame}

% \begin{frame}{}
%   \begin{definition}
%   \end{definition}
%   \begin{example}
%   \end{example}
%   \begin{block}{Exercício}
%   \end{block}
% \end{frame}

\section{Estrutura do texto}

\begin{frame}{As necessidades do leitor}
  \begin{block}{Gopen e Swan, 1990}
    ``Para o leitor apreender o que o autor quer dizer, o autor
    precisa entender o que o leitor precisa.''
  \end{block}
  \begin{block}{Hindle, 2013}
    ``Você não está escrevendo para você mesmo, está escrevendo para
    seu leitor.''
  \end{block}
  \begin{block}{Figueiredo, hoje.}
    ``O trabalho de comunicar é seu, não do leitor''.
  \end{block}
\end{frame}

\begin{frame}{Exemplo}
\end{frame}

\subsection{A posição de ênfase}

\begin{frame}{A posição de ênfase}
  \begin{example}
    \begin{itemize}
    \item Paul finalmente ganhou o prêmio depois de comprar 100
      bilhetes de loteria \alert<4->{na loja}.
    \item Paul teve que comprar 100 bilhetes de loteria até finalmente
      \alert<5->{ganhar o prêmio}.
    \end{itemize}
  \end{example}
  Fonte: Wilke, 2013.
\end{frame}

\subsection{A posição de tópico}

\begin{frame}{A posição de tópico}
  \begin{example}
    A \alert<2->{erosão do solo} é um problema sério nas regiões
    montanhosas do Nepal.
  \end{example}
  \begin{block}{Assunto}
    O assunto da frase acima é a \alert<2->{erosão do solo}.
  \end{block}
  Fonte: Hindle, 2013.
\end{frame}

\begin{frame}{A posição de tópico}
  \begin{example}
    As \alert<2->{regiões montanhosas do Nepal} encaram problemas sérios com o
    aumento da erosão do solo.
  \end{example}
  \begin{block}{Assunto}
    Esta frase indica que o assunto de interesse é \alert<2->{uma
      região específica} do Nepal.
  \end{block}
  Fonte: Hindle, 2013.
\end{frame}

\subsection{Resumo}

\begin{frame}{Princípios estruturais}

  \begin{enumerate}
  \item Posicione o verbo após o sujeito gramatical tão cedo quanto
    possível
  \item Posicione a ``nova informação'' que você quer que o leitor
    enfatize na posição de ênfase
  \item Posicione a pessoa ou coisa a que a frase se refere no início,
    na posição de tópico.
  \item Posicione qualquer ``informação antiga'' (que já foi
    discutida) na posição de tópico para relacionar com o que passou e
    contextualizar com o que virá
  \item Articule a ação de cada frase em seu verbo.
  \item Em geral, indique algum contexto para seu leitor antes de
    exigir que ele considere qualquer informação nova
  \item Em geral, tente assegurar que as ênfases relativas do conteúdo
    coincidem com as expectativas relativas levantadas pela estrutura
  \end{enumerate}
\end{frame}

\begin{frame}{Referências}
  \begin{itemize}
  \item<1-> Gopen, George; Swan, Judith. The Science of Scientific
    Writing, 1990. American Scientist.
  \item<1-> Wilke, Claus. Writing paragraphs that make sense,
    2013. \url{http://serialmentor.com/blog/2013/9/26/writing-paragraphs-that-make-sensethe-topic-and-the-stress-position/}
  \item<1-> Hindle, Amanda. Reader expectations (blog series),
    2013. \url{http://www.edanzediting.com/blog/tag/reader_expectations}
  \end{itemize}
\end{frame}
\end{document}
